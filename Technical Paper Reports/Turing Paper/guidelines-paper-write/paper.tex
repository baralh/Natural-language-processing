\documentclass[journal, a4paper]{IEEEtran}

% modified by Bridget McInnes for CS416 Introduction to NLP

% some very useful LaTeX packages include:

%\usepackage{cite}      % Written by Donald Arseneau
                        % V1.6 and later of IEEEtran pre-defines the format
                        % of the cite.sty package \cite{} output to follow
                        % that of IEEE. Loading the cite package will
                        % result in citation numbers being automatically
                        % sorted and properly "ranged". i.e.,
                        % [1], [9], [2], [7], [5], [6]
                        % (without using cite.sty)
                        % will become:
                        % [1], [2], [5]--[7], [9] (using cite.sty)
                        % cite.sty's \cite will automatically add leading
                        % space, if needed. Use cite.sty's noadjust option
                        % (cite.sty V3.8 and later) if you want to turn this
                        % off. cite.sty is already installed on most LaTeX
                        % systems. The latest version can be obtained at:
                        % http://www.ctan.org/tex-archive/macros/latex/contrib/supported/cite/

\usepackage{graphicx}   % Written by David Carlisle and Sebastian Rahtz
                        % Required if you want graphics, photos, etc.
                        % graphicx.sty is already installed on most LaTeX
                        % systems. The latest version and documentation can
                        % be obtained at:
                        % http://www.ctan.org/tex-archive/macros/latex/required/graphics/
                        % Another good source of documentation is "Using
                        % Imported Graphics in LaTeX2e" by Keith Reckdahl
                        % which can be found as esplatex.ps and epslatex.pdf
                        % at: http://www.ctan.org/tex-archive/info/

%\usepackage{psfrag}    % Written by Craig Barratt, Michael C. Grant,
                        % and David Carlisle
                        % This package allows you to substitute LaTeX
                        % commands for text in imported EPS graphic files.
                        % In this way, LaTeX symbols can be placed into
                        % graphics that have been generated by other
                        % applications. You must use latex->dvips->ps2pdf
                        % workflow (not direct pdf output from pdflatex) if
                        % you wish to use this capability because it works
                        % via some PostScript tricks. Alternatively, the
                        % graphics could be processed as separate files via
                        % psfrag and dvips, then converted to PDF for
                        % inclusion in the main file which uses pdflatex.
                        % Docs are in "The PSfrag System" by Michael C. Grant
                        % and David Carlisle. There is also some information
                        % about using psfrag in "Using Imported Graphics in
                        % LaTeX2e" by Keith Reckdahl which documents the
                        % graphicx package (see above). The psfrag package
                        % and documentation can be obtained at:
                        % http://www.ctan.org/tex-archive/macros/latex/contrib/supported/psfrag/

%\usepackage{subfigure} % Written by Steven Douglas Cochran
                        % This package makes it easy to put subfigures
                        % in your figures. i.e., "figure 1a and 1b"
                        % Docs are in "Using Imported Graphics in LaTeX2e"
                        % by Keith Reckdahl which also documents the graphicx
                        % package (see above). subfigure.sty is already
                        % installed on most LaTeX systems. The latest version
                        % and documentation can be obtained at:
                        % http://www.ctan.org/tex-archive/macros/latex/contrib/supported/subfigure/

\usepackage{url}        % Written by Donald Arseneau
                        % Provides better support for handling and breaking
                        % URLs. url.sty is already installed on most LaTeX
                        % systems. The latest version can be obtained at:
                        % http://www.ctan.org/tex-archive/macros/latex/contrib/other/misc/
                        % Read the url.sty source comments for usage information.

%\usepackage{stfloats}  % Written by Sigitas Tolusis
                        % Gives LaTeX2e the ability to do double column
                        % floats at the bottom of the page as well as the top.
                        % (e.g., "\begin{figure*}[!b]" is not normally
                        % possible in LaTeX2e). This is an invasive package
                        % which rewrites many portions of the LaTeX2e output
                        % routines. It may not work with other packages that
                        % modify the LaTeX2e output routine and/or with other
                        % versions of LaTeX. The latest version and
                        % documentation can be obtained at:
                        % http://www.ctan.org/tex-archive/macros/latex/contrib/supported/sttools/
                        % Documentation is contained in the stfloats.sty
                        % comments as well as in the presfull.pdf file.
                        % Do not use the stfloats baselinefloat ability as
                        % IEEE does not allow \baselineskip to stretch.
                        % Authors submitting work to the IEEE should note
                        % that IEEE rarely uses double column equations and
                        % that authors should try to avoid such use.
                        % Do not be tempted to use the cuted.sty or
                        % midfloat.sty package (by the same author) as IEEE
                        % does not format its papers in such ways.

\usepackage{amsmath}    % From the American Mathematical Society
                        % A popular package that provides many helpful commands
                        % for dealing with mathematics. Note that the AMSmath
                        % package sets \interdisplaylinepenalty to 10000 thus
                        % preventing page breaks from occurring within multiline
                        % equations. Use:
%\interdisplaylinepenalty=2500
                        % after loading amsmath to restore such page breaks
                        % as IEEEtran.cls normally does. amsmath.sty is already
                        % installed on most LaTeX systems. The latest version
                        % and documentation can be obtained at:
                        % http://www.ctan.org/tex-archive/macros/latex/required/amslatex/math/



% Other popular packages for formatting tables and equations include:

%\usepackage{array}
% Frank Mittelbach's and David Carlisle's array.sty which improves the
% LaTeX2e array and tabular environments to provide better appearances and
% additional user controls. array.sty is already installed on most systems.
% The latest version and documentation can be obtained at:
% http://www.ctan.org/tex-archive/macros/latex/required/tools/

% V1.6 of IEEEtran contains the IEEEeqnarray family of commands that can
% be used to generate multiline equations as well as matrices, tables, etc.

% Also of notable interest:
% Scott Pakin's eqparbox package for creating (automatically sized) equal
% width boxes. Available:
% http://www.ctan.org/tex-archive/macros/latex/contrib/supported/eqparbox/

% *** Do not adjust lengths that control margins, column widths, etc. ***
% *** Do not use packages that alter fonts (such as pslatex).         ***
% There should be no need to do such things with IEEEtran.cls V1.6 and later.


% Your document starts here!
\begin{document}

% Define document title and author
	\title{Computing Machinery and Intelligence\\A. M. Turing 1950}
	\author{Heman Baral}{}
	\maketitle
% Description of the Study
\section{Description of the Study} 

	The purpose of this research is Computing Machinery and Intelligence. Allan Turing purpose a question, ‘Can machines think?’ He says we need to define what we mean by "machines" and "think" if we want to know if machine can think, but we can't use common understanding of the definitions, because this would be committing the logical misconception since the concept and their meaning can never be define by taking the poll. 


%Methods and Design of the Study
\section{Methods and Design}
	Turing claims that his thought expirement called "Imitation game" can be used to find an answer to the question by subjecting machines. 
Suppose you have three people: a man (A), a woman (B), and an interrogator (C). C is in a separate room to A and B. Asking questiins, C's goal is to guess which of A or B is a womenm, and which is the man. A's job is to confise C, while B's job is to assist C. No matter who is the man and who is the women, both will say things like "Don't listen to him, i am the women."
Turing then asks what would if the part of A is taken by a computer instead.


% How the study was Evaluated
\section{Analysis}
	Turing thinks that if A as computer counld convince the interrogator that they are human, this functionally the same thing as saying that the computer is thinking. He considers counter example to his clain and provides evidence that these counter exaple are not convincing. The first counter is that of consciousness, something similar to the "what it feels like" because we can't enter into the inner expierence of other animals but we assume that they have it. Additionally, when you talk to other human being you think that they have a rich inner life just like yours. 
% The results of the Study
\section{Results}

Turing propose different  arguement against the possibility for conscious machines. Ultimately, though, Turing believes that such machines will eventually be able to pass intellectual test for humans. Turing also addresses the problem of creating a machine that can think and make decision like human. Instead of programming into a computer every little component of human knowledge and understanding, and the relations between, Turing argues that a program must be written that directs a computer to learn. Thus, a computer will be able to build its own understandings, just as a child does. 

% Limitations of the Study
\section{Limitations}

	Turing singles out human behavior for further analysis. Taking counter argument saying that people make mistakes all the time, but computer do not. In the limitation game, it's proposed that the interrogator would always be able to identify the computer by using complex mathematical questions. Yet, Turing says, since the computer is trying to fool the interrogator, all it would have to do is pretent to make some mistakes. Moreover, there are different kinds of error: errors of functioning and errors of conclusion. we ignore the former when we consider idealised machines, and the latter only happens when meaning is attached to the latter. The machine can't do this unless it's thinking.

% Significance of the Study
\section{Significance}

	Turing predicted that machine with 100MB of memory can easily pass the limitation game. Even though today's computer have far more memory than that only few have succeeded. Those computer which have done well fouced more on finding clever ways to fool the judges than using overwhelming computer powers. The first program some claim to success is ELIZA by mimicking a psychologist by encouring them to talk more. Since then there are lots of chatbot and robots but they have failed to mimic like human being since human language are far more complex to understand. 

% Conclusion of the Study
\section{Conclusion}
	This paper has a great impact even more advanced for computer science and engineering to crack how human being view themselves. While Alan Turing does not imply the role of the computer as a medium, I believe this article shows the benefit and importance of stepping back from the medium and asking, “How does a computer advance and mimic the way human think and senses?”


% Bibliography please include the paper that is being reviewed and any additional
% sources of information
\begin{thebibliography}{5}

	%Each item starts with a \bibitem{reference} command and the details thereafter.
	\bibitem{HOP96} % Journal paper
	J.~Hagenauer, E.~Offer, and L.~Papke. Iterative decoding of binary block
	and convolutional codes. {\em IEEE Trans. Inform. Theory},
	vol.~42, no.~2, pp.~429–-445, Mar. 1996.

	\bibitem{MJH06} % Conference paper
	T.~Mayer, H.~Jenkac, and J.~Hagenauer. Turbo base-station cooperation for intercell interference cancellation. {\em IEEE Int. Conf. Commun. (ICC)}, Istanbul, Turkey, pp.~356--361, June 2006.

	\bibitem{Proakis} % Book
	J.~G.~Proakis. {\em Digital Communications}. McGraw-Hill Book Co.,
	New York, USA, 3rd edition, 1995.

\end{thebibliography}

\end{document}