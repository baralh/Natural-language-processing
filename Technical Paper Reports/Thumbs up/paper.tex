\documentclass[journal, a4paper]{IEEEtran}

% modified by Bridget McInnes for CS416 Introduction to NLP

% some very useful LaTeX packages include:

%\usepackage{cite}      % Written by Donald Arseneau
                        % V1.6 and later of IEEEtran pre-defines the format
                        % of the cite.sty package \cite{} output to follow
                        % that of IEEE. Loading the cite package will
                        % result in citation numbers being automatically
                        % sorted and properly "ranged". i.e.,
                        % [1], [9], [2], [7], [5], [6]
                        % (without using cite.sty)
                        % will become:
                        % [1], [2], [5]--[7], [9] (using cite.sty)
                        % cite.sty's \cite will automatically add leading
                        % space, if needed. Use cite.sty's noadjust option
                        % (cite.sty V3.8 and later) if you want to turn this
                        % off. cite.sty is already installed on most LaTeX
                        % systems. The latest version can be obtained at:
                        % http://www.ctan.org/tex-archive/macros/latex/contrib/supported/cite/

\usepackage{graphicx}   % Written by David Carlisle and Sebastian Rahtz
                        % Required if you want graphics, photos, etc.
                        % graphicx.sty is already installed on most LaTeX
                        % systems. The latest version and documentation can
                        % be obtained at:
                        % http://www.ctan.org/tex-archive/macros/latex/required/graphics/
                        % Another good source of documentation is "Using
                        % Imported Graphics in LaTeX2e" by Keith Reckdahl
                        % which can be found as esplatex.ps and epslatex.pdf
                        % at: http://www.ctan.org/tex-archive/info/

%\usepackage{psfrag}    % Written by Craig Barratt, Michael C. Grant,
                        % and David Carlisle
                        % This package allows you to substitute LaTeX
                        % commands for text in imported EPS graphic files.
                        % In this way, LaTeX symbols can be placed into
                        % graphics that have been generated by other
                        % applications. You must use latex->dvips->ps2pdf
                        % workflow (not direct pdf output from pdflatex) if
                        % you wish to use this capability because it works
                        % via some PostScript tricks. Alternatively, the
                        % graphics could be processed as separate files via
                        % psfrag and dvips, then converted to PDF for
                        % inclusion in the main file which uses pdflatex.
                        % Docs are in "The PSfrag System" by Michael C. Grant
                        % and David Carlisle. There is also some information
                        % about using psfrag in "Using Imported Graphics in
                        % LaTeX2e" by Keith Reckdahl which documents the
                        % graphicx package (see above). The psfrag package
                        % and documentation can be obtained at:
                        % http://www.ctan.org/tex-archive/macros/latex/contrib/supported/psfrag/

%\usepackage{subfigure} % Written by Steven Douglas Cochran
                        % This package makes it easy to put subfigures
                        % in your figures. i.e., "figure 1a and 1b"
                        % Docs are in "Using Imported Graphics in LaTeX2e"
                        % by Keith Reckdahl which also documents the graphicx
                        % package (see above). subfigure.sty is already
                        % installed on most LaTeX systems. The latest version
                        % and documentation can be obtained at:
                        % http://www.ctan.org/tex-archive/macros/latex/contrib/supported/subfigure/

\usepackage{url}        % Written by Donald Arseneau
                        % Provides better support for handling and breaking
                        % URLs. url.sty is already installed on most LaTeX
                        % systems. The latest version can be obtained at:
                        % http://www.ctan.org/tex-archive/macros/latex/contrib/other/misc/
                        % Read the url.sty source comments for usage information.

%\usepackage{stfloats}  % Written by Sigitas Tolusis
                        % Gives LaTeX2e the ability to do double column
                        % floats at the bottom of the page as well as the top.
                        % (e.g., "\begin{figure*}[!b]" is not normally
                        % possible in LaTeX2e). This is an invasive package
                        % which rewrites many portions of the LaTeX2e output
                        % routines. It may not work with other packages that
                        % modify the LaTeX2e output routine and/or with other
                        % versions of LaTeX. The latest version and
                        % documentation can be obtained at:
                        % http://www.ctan.org/tex-archive/macros/latex/contrib/supported/sttools/
                        % Documentation is contained in the stfloats.sty
                        % comments as well as in the presfull.pdf file.
                        % Do not use the stfloats baselinefloat ability as
                        % IEEE does not allow \baselineskip to stretch.
                        % Authors submitting work to the IEEE should note
                        % that IEEE rarely uses double column equations and
                        % that authors should try to avoid such use.
                        % Do not be tempted to use the cuted.sty or
                        % midfloat.sty package (by the same author) as IEEE
                        % does not format its papers in such ways.

\usepackage{amsmath}    % From the American Mathematical Society
                        % A popular package that provides many helpful commands
                        % for dealing with mathematics. Note that the AMSmath
                        % package sets \interdisplaylinepenalty to 10000 thus
                        % preventing page breaks from occurring within multiline
                        % equations. Use:
%\interdisplaylinepenalty=2500
                        % after loading amsmath to restore such page breaks
                        % as IEEEtran.cls normally does. amsmath.sty is already
                        % installed on most LaTeX systems. The latest version
                        % and documentation can be obtained at:
                        % http://www.ctan.org/tex-archive/macros/latex/required/amslatex/math/



% Other popular packages for formatting tables and equations include:

%\usepackage{array}
% Frank Mittelbach's and David Carlisle's array.sty which improves the
% LaTeX2e array and tabular environments to provide better appearances and
% additional user controls. array.sty is already installed on most systems.
% The latest version and documentation can be obtained at:
% http://www.ctan.org/tex-archive/macros/latex/required/tools/

% V1.6 of IEEEtran contains the IEEEeqnarray family of commands that can
% be used to generate multiline equations as well as matrices, tables, etc.

% Also of notable interest:
% Scott Pakin's eqparbox package for creating (automatically sized) equal
% width boxes. Available:
% http://www.ctan.org/tex-archive/macros/latex/contrib/supported/eqparbox/

% *** Do not adjust lengths that control margins, column widths, etc. ***
% *** Do not use packages that alter fonts (such as pslatex).         ***
% There should be no need to do such things with IEEEtran.cls V1.6 and later.


% Your document starts here!
\begin{document}

% Define document title and author
	\title{Thumbs Up or Thumbs Down? Semantic Orientation Applied to Unsupervised Classification of Reviews\\Peter D. Turney}
	\author{Heman Baral}{}
	\maketitle
% Description of the Study
\section{Description of the Study} 

The purpose of this study is evaluating semantic orientation. The average semantic orientation of the phrases contains adjectives or adverbs. A phrase has a positive semantic orientation when it has good associations and a negative semantic orientation when it has bad associations. The semantic orientation of a phrase is calculated as the mutual information between the given phrase and the word excellent minus the mutual information between the given phrase and the word poor. By looking at Epinions reviews of automobiles, banks, movies, and travel destinations The algorithm achieves an average accuracy for recommended.


%Methods and Design of the Study
\section{Methods and Design}
	In a search engine, we search for some information in the query expected information and found the matches. It divides the query into fractions and matches recommend phrase. Search engines report the summary data with an unsupervised learning algorithm. The algorithm takes a written review as input and produces a classification as output. This algorithm first uses part-of-speech tagger to identify phrases in the input text that contain adjectives or adverbs then estimate the semantic orientation of each extracted phrase. A phrase has a positive and negative semantic orientation. Based on the average semantic orientation of the phrases extracted from the review by assigning a class. If the average is positive then review recommends the item otherwise the item is not recommended. The PMI-IR algorithm estimates the semantic orientation of a phrase and measures the similarity of pairs of words or phrases. A phrase is assigned a numerical rating by taking the mutual information between the given phrase. The word excellent and subtracting the mutual information between the given phrase and the word poor. The direction of the phrases semantic orientation positive or negative, based on the sign of the rating. Hatzivassiloglou and McKeown have also developed an algorithm for predicting semantic orientation. Designed for isolated adjectives, rather than phrases containing adjectives or adverbs.


% How the study was Evaluated
\section{Analysis}
	The PMI-IR algorithm extracts two consecutive words, where one member of the pair is an adjective or an adverb and the second provides context. Comparing its similarity to a positive reference word excellent and negative reference word poor
calculate the semantic orientation of a given phrase.positive or negative numerical rating indicates the strength of the semantic orientation based on the magnitude of the number. Hatzivassiloglou and McKeown use a four-step supervised learning algorithm to infer the semantic orientation of adjectives from constraints on conjunctions. All conjunctions of adjectives are extracted from the given corpus and supervised learning algorithm combines multiple sources of evidence to label pairs of adjectives as having the same or different semantic orientation. A clustering algorithm processes the graph structure to produce two subsets of adjectives. Mainly different-orientation links and links inside a subset are mainly same-orientation links. Positive adjectives tend to be used more frequently than negative adjectives, the cluster with the higher average frequency is classified as having a positive semantic orientation.

% The results of the Study
\section{Results}

The average Semantic Orientation recommends by the review. The classification algorithm is evaluated on 410 reviews from Epinions, randomly sampled from the domains of automobiles, banks, movies, and travel destinations. From these reviews, 170 are not recommended and the remaining 240 are recommended. The algorithm achieves an average accuracy of 74 percent while the baseline of accuracy is 59 percent. The clustering algorithm classifies adjectives with accuracies ranging from 78 percent to 92 percent, depending on the amount of training data that is available.

% Limitations of the Study
\section{Limitations}

	The limitations of this work include the time required for queries and the level of accuracy that was achieved. Thumbs up or down system for generating sentiment timeline tracks online discussions about movies and displays a plot of the number of positive sentiment and negative sentiment messages over time but messages are classified by looking for specific phrases that indicate the sentiment of the author towards the movie. The movie aspect is not the same in every place and also they took a smaller sample size. An isolated adjective may be insufficient context to determine semantic orientation. Based on the presence or absence of specific words could supplement average semantic orientation in a supervised classification system and it could yield higher accuracies.

% Significance of the Study
\section{Significance}

	The review classification is based on an average which is might be quite resistant to noise in the semantic orientation estimate for individual phrases but a semantic orientation estimator can produce a better classification. It might benefit from more sophisticated statistical analysis to apply a statistical significance test to each estimated SO. There is a large statistical literature on the log-odds ratio, which might lead to improved results on this task. In this section discuss the significance of the work.
% Conclusion of the Study
\section{Conclusion}
	A simple unsupervised learning algorithm takes a written review as input and produces a classification as output which makes our life easy. The searching engine divided into parts-of-speech and identify the target. For a small part of information, it followed semantic orientation and divided into good and bad associations. The PMI-IR algorithm is employed to estimate the semantic orientation of a phrase and given phrase is calculated by comparing its similarity to a positive reference word excellent and negative word poor. Hatzivassiloglou and McKeown have also developed an algorithm for predicting semantic orientation. Their algorithm performs well, but it is designed for isolated adjectives, rather than phrases containing adjectives or adverbs. The semantic orientation is calculated using the natural logarithm base e, rather than base 2.


% Bibliography please include the paper that is being reviewed and any additional
% sources of information
\begin{thebibliography}{5}

	%Each item starts with a \bibitem{reference} command and the details thereafter.
	\bibitem{HOP96} % Journal paper
Thumbs Up or Thumbs Down? Semantic Orientation Applied to Unsupervised Classification of Reviews. [Cs/0212032] Thumbs Up or Thumbs Down? Semantic Orientation Applied to Unsupervised Classification of Reviews, Peter D. Turney, 11 Dec. 2002, arxiv.org/abs/cs/0212032..


\end{thebibliography}

\end{document}