\documentclass[journal, a4paper]{IEEEtran}

% modified by Bridget McInnes for CS416 Introduction to NLP

% some very useful LaTeX packages include:

%\usepackage{cite}      % Written by Donald Arseneau
                        % V1.6 and later of IEEEtran pre-defines the format
                        % of the cite.sty package \cite{} output to follow
                        % that of IEEE. Loading the cite package will
                        % result in citation numbers being automatically
                        % sorted and properly "ranged". i.e.,
                        % [1], [9], [2], [7], [5], [6]
                        % (without using cite.sty)
                        % will become:
                        % [1], [2], [5]--[7], [9] (using cite.sty)
                        % cite.sty's \cite will automatically add leading
                        % space, if needed. Use cite.sty's noadjust option
                        % (cite.sty V3.8 and later) if you want to turn this
                        % off. cite.sty is already installed on most LaTeX
                        % systems. The latest version can be obtained at:
                        % http://www.ctan.org/tex-archive/macros/latex/contrib/supported/cite/

\usepackage{graphicx}   % Written by David Carlisle and Sebastian Rahtz
                        % Required if you want graphics, photos, etc.
                        % graphicx.sty is already installed on most LaTeX
                        % systems. The latest version and documentation can
                        % be obtained at:
                        % http://www.ctan.org/tex-archive/macros/latex/required/graphics/
                        % Another good source of documentation is "Using
                        % Imported Graphics in LaTeX2e" by Keith Reckdahl
                        % which can be found as esplatex.ps and epslatex.pdf
                        % at: http://www.ctan.org/tex-archive/info/

%\usepackage{psfrag}    % Written by Craig Barratt, Michael C. Grant,
                        % and David Carlisle
                        % This package allows you to substitute LaTeX
                        % commands for text in imported EPS graphic files.
                        % In this way, LaTeX symbols can be placed into
                        % graphics that have been generated by other
                        % applications. You must use latex->dvips->ps2pdf
                        % workflow (not direct pdf output from pdflatex) if
                        % you wish to use this capability because it works
                        % via some PostScript tricks. Alternatively, the
                        % graphics could be processed as separate files via
                        % psfrag and dvips, then converted to PDF for
                        % inclusion in the main file which uses pdflatex.
                        % Docs are in "The PSfrag System" by Michael C. Grant
                        % and David Carlisle. There is also some information
                        % about using psfrag in "Using Imported Graphics in
                        % LaTeX2e" by Keith Reckdahl which documents the
                        % graphicx package (see above). The psfrag package
                        % and documentation can be obtained at:
                        % http://www.ctan.org/tex-archive/macros/latex/contrib/supported/psfrag/

%\usepackage{subfigure} % Written by Steven Douglas Cochran
                        % This package makes it easy to put subfigures
                        % in your figures. i.e., "figure 1a and 1b"
                        % Docs are in "Using Imported Graphics in LaTeX2e"
                        % by Keith Reckdahl which also documents the graphicx
                        % package (see above). subfigure.sty is already
                        % installed on most LaTeX systems. The latest version
                        % and documentation can be obtained at:
                        % http://www.ctan.org/tex-archive/macros/latex/contrib/supported/subfigure/

\usepackage{url}        % Written by Donald Arseneau
                        % Provides better support for handling and breaking
                        % URLs. url.sty is already installed on most LaTeX
                        % systems. The latest version can be obtained at:
                        % http://www.ctan.org/tex-archive/macros/latex/contrib/other/misc/
                        % Read the url.sty source comments for usage information.

%\usepackage{stfloats}  % Written by Sigitas Tolusis
                        % Gives LaTeX2e the ability to do double column
                        % floats at the bottom of the page as well as the top.
                        % (e.g., "\begin{figure*}[!b]" is not normally
                        % possible in LaTeX2e). This is an invasive package
                        % which rewrites many portions of the LaTeX2e output
                        % routines. It may not work with other packages that
                        % modify the LaTeX2e output routine and/or with other
                        % versions of LaTeX. The latest version and
                        % documentation can be obtained at:
                        % http://www.ctan.org/tex-archive/macros/latex/contrib/supported/sttools/
                        % Documentation is contained in the stfloats.sty
                        % comments as well as in the presfull.pdf file.
                        % Do not use the stfloats baselinefloat ability as
                        % IEEE does not allow \baselineskip to stretch.
                        % Authors submitting work to the IEEE should note
                        % that IEEE rarely uses double column equations and
                        % that authors should try to avoid such use.
                        % Do not be tempted to use the cuted.sty or
                        % midfloat.sty package (by the same author) as IEEE
                        % does not format its papers in such ways.

\usepackage{amsmath}    % From the American Mathematical Society
                        % A popular package that provides many helpful commands
                        % for dealing with mathematics. Note that the AMSmath
                        % package sets \interdisplaylinepenalty to 10000 thus
                        % preventing page breaks from occurring within multiline
                        % equations. Use:
%\interdisplaylinepenalty=2500
                        % after loading amsmath to restore such page breaks
                        % as IEEEtran.cls normally does. amsmath.sty is already
                        % installed on most LaTeX systems. The latest version
                        % and documentation can be obtained at:
                        % http://www.ctan.org/tex-archive/macros/latex/required/amslatex/math/



% Other popular packages for formatting tables and equations include:

%\usepackage{array}
% Frank Mittelbach's and David Carlisle's array.sty which improves the
% LaTeX2e array and tabular environments to provide better appearances and
% additional user controls. array.sty is already installed on most systems.
% The latest version and documentation can be obtained at:
% http://www.ctan.org/tex-archive/macros/latex/required/tools/

% V1.6 of IEEEtran contains the IEEEeqnarray family of commands that can
% be used to generate multiline equations as well as matrices, tables, etc.

% Also of notable interest:
% Scott Pakin's eqparbox package for creating (automatically sized) equal
% width boxes. Available:
% http://www.ctan.org/tex-archive/macros/latex/contrib/supported/eqparbox/

% *** Do not adjust lengths that control margins, column widths, etc. ***
% *** Do not use packages that alter fonts (such as pslatex).         ***
% There should be no need to do such things with IEEEtran.cls V1.6 and later.


% Your document starts here!
\begin{document}

% Define document title and author
	\title{Part-of-Speech Tagging for Twitter: Annotation,
Features, and Experiments\\Gimpel 2011}
	\author{Heman Baral}{}
	\maketitle
% Description of the Study
\section{Description of the Study} 

	The purpose of this research is about creating POS tagging by using tweeter data. Creating parts of speech tagging on tweeter data was difficult then Standard English text because of the text on tweeter is informal and 140 character limit(280 character limit now) of each message. By using metaphone algorithm and adding some future on the parts of speech tagger the researcher team develop a tag set that provide 90 percent accuracy.


%Methods and Design of the Study
\section{Methods and Design}
	There were 17 people works together for this project and develop a tree banks by categories twitter using URL and hash tag and then they tokenized random sample of American English tweets data by using twitter tokenizer. In order to speed up the annotation researcher team pre-tagged them by using the WSJ-trained Stanford POS Tagger. A twitter word was categories in tokens by taking precedence over the Stanford tags.They distributed tweets to the annotators and classified non-English word and removed them. The annotation process wasnt cover every situation on the tweet text and for that reason the tagset, annotation guidelines, and tokenization rules were poor or ambiguous. First two annotators examined and fixed all of the English tweets tagged. By reading the annotation instructions the third annotator estimating inter-annotator agreement to create tweets and compare it with tagged tokens. Another annotator correct errors and improve consistency of tagging decisions across the corpus and those data gives the output.


% How the study was Evaluated
\section{Analysis}
	Twitter hashtags, twitter at-mentions, re-tweet,URLs, emoti- cons, hashtags and ellipsis dot makes harder to separating words. They used regular expression to separating word and comparing those word with sentence and find the most frequency word and provide the right parts of speech token. For the short from words they combine the preprocess nominal, pronoun, verb and possessive. Partial words, artifacts of to kenization errors, miscellaneous symbols, possessive endings multiword abbreviations and arrows that are not used as discourse markers that do not fit in any of the other they classified in G tagger. Suffix and capitalization patterns unpredictable in the word therefore they added those words in gazetteers tokens. They build a tagger to checks each of the word up to length 3 and separate mostly tags in the words and see what they project. Because of variety word in twitter they use methaphone algorithm which contains of 19 rules to match similar words and names in sentence and provide same key for them.To provide same key they rewrite consonants and delete vowel. They also provide token for most frequent tag for PTB words. Then they evaluated the the parts of speech tagger system that they created by metaphone algorithm and compare with Standford tagger.

% The results of the Study
\section{Results}

They chose randomly twietter data for parts of speech tagging from 1.9 million token from 134,00 unlabled tweet and divided 1,827 annotated tweets into a training set of 1,000 and develop a set of 327 tokens , and a test set of 500 and compare our system against the Stanford tagger and the found out tagging result nearly 90 percent correct. 

% Limitations of the Study
\section{Limitations}

	The part-of-speech tagging data was limited training and the tagger struggles to identify proper nouns with nonstan- dard capitalization.The annotation process doesnot covered all the situations for tagset and tokenization rules wasn’t suffi- cent.They tokenize word shape in a 3-word window because of that the 5-word shape word wasnt feet on the tagger set as a result they couldnt compare data with pre trained Stanford tagger to their data.

% Significance of the Study
\section{Significance}

	The tagger with full feature wasn’t able to reduce error 100 percent but was able to reduce 25 percent on standford tag- ger.The underline tokens were also incorrect in some specific condition and elects, governor and next were providing wrong tagger for those word in twitter.Within was also misclassified with ohh.Additionally,shoutout was appearance one time and identify as verb and rare token that provided in G token also provide errors.

% Conclusion of the Study
\section{Conclusion}
	The success of this approach demonstrates that with careful design, supervised machine learning can be applied to rapidly produce effective language technology in new domains. The data and tools have been made available to the research community with the goal of enabling richer text analysis of Twitter and related social media data sets.


% Bibliography please include the paper that is being reviewed and any additional
% sources of information
\begin{thebibliography}{5}

	%Each item starts with a \bibitem{reference} command and the details thereafter.
	\bibitem{HOP96} % Journal paper
	Brendan O’Connor, Ramnath Balasubramanyan, Bryan R. Routledge, and Noah A. Smith. 2010a. From tweets to polls: Linking text sentiment to public opinion time series. In Proc. of ICWSM.

	\bibitem{MJH06} % Conference paper
	Tim Finin, Will Murnane, Anand Karandikar, Nicholas Keller, Justin Martineau, and Mark Dredze. 2010. An- notating named entities in Twitter data with crowd- sourcing. In Proceedings of the NAACL HLT 2010 Workshop on Creating Speech and Language Data with Amazon’s Mechanical Turk.

	\bibitem{Proakis} % Book
	Grady Ward. 1996. Moby lexicon. http://icon. shef.ac.uk/Moby.

\end{thebibliography}

\end{document}