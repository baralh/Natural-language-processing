\documentclass[journal, a4paper]{IEEEtran}

% modified by Bridget McInnes for CS416 Introduction to NLP

% some very useful LaTeX packages include:

%\usepackage{cite}      % Written by Donald Arseneau
                        % V1.6 and later of IEEEtran pre-defines the format
                        % of the cite.sty package \cite{} output to follow
                        % that of IEEE. Loading the cite package will
                        % result in citation numbers being automatically
                        % sorted and properly "ranged". i.e.,
                        % [1], [9], [2], [7], [5], [6]
                        % (without using cite.sty)
                        % will become:
                        % [1], [2], [5]--[7], [9] (using cite.sty)
                        % cite.sty's \cite will automatically add leading
                        % space, if needed. Use cite.sty's noadjust option
                        % (cite.sty V3.8 and later) if you want to turn this
                        % off. cite.sty is already installed on most LaTeX
                        % systems. The latest version can be obtained at:
                        % http://www.ctan.org/tex-archive/macros/latex/contrib/supported/cite/

\usepackage{graphicx}   % Written by David Carlisle and Sebastian Rahtz
                        % Required if you want graphics, photos, etc.
                        % graphicx.sty is already installed on most LaTeX
                        % systems. The latest version and documentation can
                        % be obtained at:
                        % http://www.ctan.org/tex-archive/macros/latex/required/graphics/
                        % Another good source of documentation is "Using
                        % Imported Graphics in LaTeX2e" by Keith Reckdahl
                        % which can be found as esplatex.ps and epslatex.pdf
                        % at: http://www.ctan.org/tex-archive/info/

%\usepackage{psfrag}    % Written by Craig Barratt, Michael C. Grant,
                        % and David Carlisle
                        % This package allows you to substitute LaTeX
                        % commands for text in imported EPS graphic files.
                        % In this way, LaTeX symbols can be placed into
                        % graphics that have been generated by other
                        % applications. You must use latex->dvips->ps2pdf
                        % workflow (not direct pdf output from pdflatex) if
                        % you wish to use this capability because it works
                        % via some PostScript tricks. Alternatively, the
                        % graphics could be processed as separate files via
                        % psfrag and dvips, then converted to PDF for
                        % inclusion in the main file which uses pdflatex.
                        % Docs are in "The PSfrag System" by Michael C. Grant
                        % and David Carlisle. There is also some information
                        % about using psfrag in "Using Imported Graphics in
                        % LaTeX2e" by Keith Reckdahl which documents the
                        % graphicx package (see above). The psfrag package
                        % and documentation can be obtained at:
                        % http://www.ctan.org/tex-archive/macros/latex/contrib/supported/psfrag/

%\usepackage{subfigure} % Written by Steven Douglas Cochran
                        % This package makes it easy to put subfigures
                        % in your figures. i.e., "figure 1a and 1b"
                        % Docs are in "Using Imported Graphics in LaTeX2e"
                        % by Keith Reckdahl which also documents the graphicx
                        % package (see above). subfigure.sty is already
                        % installed on most LaTeX systems. The latest version
                        % and documentation can be obtained at:
                        % http://www.ctan.org/tex-archive/macros/latex/contrib/supported/subfigure/

\usepackage{url}        % Written by Donald Arseneau
                        % Provides better support for handling and breaking
                        % URLs. url.sty is already installed on most LaTeX
                        % systems. The latest version can be obtained at:
                        % http://www.ctan.org/tex-archive/macros/latex/contrib/other/misc/
                        % Read the url.sty source comments for usage information.

%\usepackage{stfloats}  % Written by Sigitas Tolusis
                        % Gives LaTeX2e the ability to do double column
                        % floats at the bottom of the page as well as the top.
                        % (e.g., "\begin{figure*}[!b]" is not normally
                        % possible in LaTeX2e). This is an invasive package
                        % which rewrites many portions of the LaTeX2e output
                        % routines. It may not work with other packages that
                        % modify the LaTeX2e output routine and/or with other
                        % versions of LaTeX. The latest version and
                        % documentation can be obtained at:
                        % http://www.ctan.org/tex-archive/macros/latex/contrib/supported/sttools/
                        % Documentation is contained in the stfloats.sty
                        % comments as well as in the presfull.pdf file.
                        % Do not use the stfloats baselinefloat ability as
                        % IEEE does not allow \baselineskip to stretch.
                        % Authors submitting work to the IEEE should note
                        % that IEEE rarely uses double column equations and
                        % that authors should try to avoid such use.
                        % Do not be tempted to use the cuted.sty or
                        % midfloat.sty package (by the same author) as IEEE
                        % does not format its papers in such ways.

\usepackage{amsmath}    % From the American Mathematical Society
                        % A popular package that provides many helpful commands
                        % for dealing with mathematics. Note that the AMSmath
                        % package sets \interdisplaylinepenalty to 10000 thus
                        % preventing page breaks from occurring within multiline
                        % equations. Use:
%\interdisplaylinepenalty=2500
                        % after loading amsmath to restore such page breaks
                        % as IEEEtran.cls normally does. amsmath.sty is already
                        % installed on most LaTeX systems. The latest version
                        % and documentation can be obtained at:
                        % http://www.ctan.org/tex-archive/macros/latex/required/amslatex/math/



% Other popular packages for formatting tables and equations include:

%\usepackage{array}
% Frank Mittelbach's and David Carlisle's array.sty which improves the
% LaTeX2e array and tabular environments to provide better appearances and
% additional user controls. array.sty is already installed on most systems.
% The latest version and documentation can be obtained at:
% http://www.ctan.org/tex-archive/macros/latex/required/tools/

% V1.6 of IEEEtran contains the IEEEeqnarray family of commands that can
% be used to generate multiline equations as well as matrices, tables, etc.

% Also of notable interest:
% Scott Pakin's eqparbox package for creating (automatically sized) equal
% width boxes. Available:
% http://www.ctan.org/tex-archive/macros/latex/contrib/supported/eqparbox/

% *** Do not adjust lengths that control margins, column widths, etc. ***
% *** Do not use packages that alter fonts (such as pslatex).         ***
% There should be no need to do such things with IEEEtran.cls V1.6 and later.


% Your document starts here!
\begin{document}

% Define document title and author
	\title{Vocabulary Changes in Agatha Christies Mysteries as an Indication of Dementia
Ian Lancashire and Graeme Hirst}
	\author{Heman Baral}{}
	\maketitle
% Description of the Study
\section{Description of the Study} 
The main purpose of this research is about Dementia and how it affects the brain. Dementia is a collection of symptoms which is caused by a number of disorders that affect the human brain. It is significantly decreased intellectual functioning that hampers human normal activities and relationships.Alzheimers disease is one of them disorder that leads to changes in language production at all levels. Ian Lancashire and Graeme Hirst, Professor of the University Toronto, try to recognize Alzheimers disease by analyzing Agatha Christie’s Novels.The British mystery writer Agatha Christie, who believed to have suffered from dementia in her final years when she was writing her last novel.

%Methods and Design of the Study
\section{Methods and Design}
They examine Alzheimers disease, by comparing her work from early ages to the end ages.They simple measure of vocabulary size and richness by counting the number of different words writer used. They counted the number of different maximal phrasetypes that were repeated in her novel. These are defined by word length and frequency. They counted the number of occurrences of the vague, indefinite words thing, anything, and something. Subsequently after all punctuation, apostrophes, and hyphens were deleted afterward each text was divided into 10,000-word segments.

% How the study was Evaluated
\section{Analysis}
Segments data was used to evaluated with the software tools Concordance and the Text Analysis Computing Tools (TACT). Study shows that at the age of her 81, she had a incredible drop in vocabulary compared with her other novel that was written 18 years earlier.In addition, their analysis suggests that repeating phrases and using indefinite term are significant markers for dementia.

% The results of the Study
\section{Results}

Furthermore Christies age increases, her vocabulary of the novel declines the productivity. In her novels, the number of different repeating phrase types in the first 50,000 words increases with age and highly decline the lexical richness of her writing. The three novels that she wrote in her 80s, Nemesis, Elephants, and Postern, have a smaller vocabulary than any of her novel written by her between ages 28 to 63.

% Limitations of the Study
\section{Limitations}

Professor lancashire and hirst,analyze a selection of Christies novels but not all of her novels. Furthermore, they just took first 50,000 words from each novel when her novel contain between 55,000 and 75,000 words. They identified how often she relied on indefinite words, such as thing, anything and something, rather than more specific words. Further more, Agatha Christie never diagnosed by doctors for dementia. By the time of Elephants can Remember, Christie was older and broken a hip. Possibly that could be a fact that she was unable to create a crime solvable by clue detection .


% Significance of the Study
\section{Significance}

Owing to Dementia, in her last novels she wasn’t able to create a crime solvable by clue detection, according to the rules of the genre that she create. Normally her detective novel used to gives readers a problem to be solved. She used to provide clues and maintains confusion until her sleuth exposes a mysterious solution that ends the book. Her tradition, was to work out the plot strictly beforehand in a notebook, and to write the last chapter where her detective laid out the solution first. But her preoccupation with old people and their memories in both Elephants can Remember and Postern of Fate reflects more on her personal circumstances than on crime, murderer, and clues.

% Conclusion of the Study
\section{Conclusion}
	Alzheimer’s disease cause dementia .It is starts slowly and get worse over time. The most common early symptom is difficulty in remembering recent events. It causes short-term memory loss in early symptom and in advances level patient can face language problems, confusion, mood swings, loss of motivation, and behavioral issues.Christies inconsistencies in character and plotting in both these late works. Her readers complained about her and the agent directed for editorial help. As a result her husband, secretary and daughter request the press to not print any more books. we can conclude that , Physical and mental decline is sad.As a person’s condition falls, they often withdraw from family and society.

% Bibliography please include the paper that is being reviewed and any additional
% sources of information
\begin{thebibliography}{5}

	%Each item starts with a \bibitem{reference} command and the details thereafter.
	\bibitem{HOP96} % Journal paper
	Flood, Alison. “Study Claims Agatha Christie Had Alzheimer's.” The Guardian, Guardian News and Media, 3 Apr. 2009, www.theguardian.com/books/2009/apr/03/agatha-christie-alzheimers-research.

	\bibitem
Kate Devlin, Medical Correspondent. “Agatha Christie 'Had Alzheimer's Disease When She Wrote Final Novels'.” The Telegraph, Telegraph Media Group, 4 Apr. 2009, www.telegraph.co.uk/news/health/news/5101619/Agatha-Christie-had-Alzheimers-disease-when-she-wrote-final-novels.html.
\end{thebibliography}





\end{document}